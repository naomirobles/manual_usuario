\documentclass{Pretexto/bluereport}

\title{Manual de usuario}
\author{}
\date{}

\begin{document}

\begin{titlepage}
    \thispagestyle{empty}
    
    \pagecolor{white}
    
    % Ahora dibujamos las bandas laterales con \rule{}
    \begin{picture}(0,0)
        % Primera banda (amarilla, la más a la izquierda)
        \put(-94,-800){\textcolor{yellowdark}{\rule{3cm}{99.7cm}}} 
        % Segunda banda (verde)
        \put(-48,-800){\textcolor{yellowlight}{\rule{1.5cm}{99.7cm}}}
        % Tercera banda (azul)
        \put(-20,-800){\textcolor{primaryblue}{\rule{1cm}{99.7cm}}}
    \end{picture}
    
    % Contenido principal de la portada
    \begin{flushleft}
        \vspace*{1cm}

        % Ajusta este valor para mover todo hacia la derecha
        \hspace*{4.5cm}
        \begin{minipage}{13cm} % ancho de bloque de texto
            % Logo 

            \includegraphics[width=6cm]{img/logo.png}


            \vspace{3cm} 
            
            % Títulos y texto
            {\fontsize{24}{28}\selectfont\color{primaryblue}
            \textbf{COTIZADOR LA LIGA}\par}

            \vspace{2cm}

            {\fontsize{36}{42}\selectfont\color{primaryblue}
            \textbf{Manual de Usuario}\par}
            
            \vspace{1cm}
            
            {\fontsize{16}{20}\selectfont\color{charcoal}
            Versión 1.0\par}
            
            \vfill 
            
            \rule{10cm}{1pt}
            
            \vspace{0.5cm}
            
            {\fontsize{16}{20}\selectfont\color{charcoal}
            Septiembre 2025\par}
        \end{minipage}
    \end{flushleft}
\end{titlepage}


\tableofcontents
\pagebreak

%%%%%%%%%%%%%%%%%%%%%%%%%%%%%%%%%%%%%%%%%%%%%%%%%%%%
\section{Introducción}


%%%%%%%%%%%%%%%%%%%%%%%%%%%%%%%%%%%%%%%%%%%%%%%%%%%%%%%%%%%%%%%%%%%%%%%%%%%%%%%%%%%%
\section{Requisitos del Sistema}

\subsection{Requisitos Mínimos de Hardware}
\begin{center}
\begin{tcolorbox}[
    enhanced,
    boxrule=2pt,
    colframe=primaryblue,
    colback=light,
    rounded corners=5pt,
    width=0.98\textwidth
]

\renewcommand{\arraystretch}{1.6}
\begin{tabular}{>{\raggedright\bfseries}p{3.5cm} 
                >{\raggedright\arraybackslash}p{9.5cm}}

\textcolor{primaryblue}{Procesador:} & 
\begin{minipage}[t]{9cm}
\vspace{-0.3cm}
\begin{itemize}[leftmargin=10pt, itemsep=2pt]
    \item \textcolor{black}{Intel Core i3 de 4ta generación o AMD equivalente}
    \item \textcolor{black}{Arquitectura x64 (64 bits)}
    \item \textcolor{black}{\textbf{Velocidad mínima de 2.0 GHz}}
\end{itemize}
\vspace{0.7cm}
\end{minipage} \\[8pt]
\hline
\textcolor{primaryblue}{Memoria RAM:} & 
\begin{minipage}[t]{9cm}
\vspace{-0.3cm}
\begin{itemize}[leftmargin=10pt, itemsep=2pt]
    \item \textcolor{black}{4 GB de RAM como mínimo}
    \item \textcolor{black}{\textbf{8 GB recomendado para mejor rendimiento}}
    \item \textcolor{black}{Velocidad mínima de 2.0 GHz}
\end{itemize}
\vspace{0.7cm}
\end{minipage} \\[8pt]
\hline
\textcolor{primaryblue}{Almacenamiento:} & 
\begin{minipage}[t]{9cm}
\vspace{-0.3cm}
\begin{itemize}[leftmargin=10pt, itemsep=2pt]
    \item \textcolor{black}{500 MB de espacio libre en disco duro para la instalación base}
    \item \textcolor{black}{200 MB adicionales para base de datos y archivos temporales}
    \item \textcolor{black}{\textbf{Espacio adicional según el volumen de cotizaciones almacenadas}}
\end{itemize}
\vspace{0.7cm}
\end{minipage} \\[8pt]
\hline
\textcolor{primaryblue}{Otros componentes:} & 
\textcolor{black}{\textbf{Conexión a internet recomendada.}} \\[5pt]

\end{tabular}
\end{tcolorbox}
\end{center}

%%%%%%%%%%%%%%%%%%%%%%%%%%%%%%%%%%%%%%%%%%%%%%%%%%%%%%%%%%%%%%%%%%%%%%%%%%%%%%%%%%

\subsection{Requisitos de Software}
\begin{itemize}
    \item Lector de archivos PDF (para visualizar las cotizaciones generadas)
    \item Microsoft Excel o software compatible para crear archivos .xlsx/.xls (opcional, solo si se utilizará la función de importación)
\end{itemize}

\subsection{Sistemas operativos compatibles.}
\begin{itemize}
    \item Windows 10 (versión 1903 o superior)
    \item Windows 11 (todas las versiones)
\end{itemize}

Arquitectura del sistema: x64 (64 bits), no compatible con sistemas operativos de 32 bits.

\subsubsection{Notas adicionales}
\begin{itemize}
    \item La aplicación incluye todas las dependencias necesarias (SQLite, Puppeteer) en el paquete de instalación.
    \item No requiere instalación de Node.js ni otras herramientas de desarrollo.
    \item El archivo de base de datos se almacena localmente y no requiere servidor externo.
\end{itemize}


%%%%%%%%%%%%%%%%%%%%%%%%%%%%%%%%%%%%%%%%%%%%%%%%%%%%
\section{Acceso}
\subsection{Cómo abrir la aplicaión}

%%%%%%%%%%%%%%%%%%%%%%%%%%%%%%%%%%%%%%%%%%%%%%%%%%%%
\section{Interaz Principal y Funcionalidades}
\subsection{Explicación del menú de Gestión de Cotizaciones}
\subsection{Página para agregar cotizaciones}
\subsubsection{Importar desde Excel}
Importar desde Excel
\subsubsection{Gestión de productos/servicios}
Gestión de productos/servicios
\subsubsection{Opciones de guardado}

%%%%%%%%%%%%%%%%%%%%%%%%%%%%%%%%%%%%%%%%%%%%%%%%%%%%
\section{Ejemplos prácticos}
\subsection{Cómo registrar una cotización}
Al abrir por primera vez la aplicación, la página de \textbf{Cotizaciones guardadas} se verá como se muestra en la figura 1.
Como aun no hay registros de cotizaciones, la tabla estará vacía. Para agregar una nueva cotización, haz clic en el botón \textbf{Agregar Cotización} o dentro de
la tabla en \textbf{Crear primera cotización}. Esto te llevará a la página para agregar una nueva cotización, donde podrás ingresar los detalles necesarios.
\begin{figure}[H]
    \centering
        \includegraphics[width=0.85\linewidth]{img/abrir_primera_vez.png}
    \caption{Vista inicial de la página de cotizaciones guardadas al abrir la aplicación por primera vez.}
    \label{fig:vista_inicial}
\end{figure}
Una vez en la página de agregar cotización, podrás ver la siguiente interfaz (figura 2). Aquí podrás ingresar los detalles de la cotización.
\begin{figure}[H] 
    \centering
        \includegraphics[width=0.85\linewidth]{img/add_cotizacion_inicial.png}
    \caption{Vista inicial de la página de agregar cotización al abrir la aplicación por primera vez.}
    \label{fig:vista_inicial_add}
\end{figure}
En esta parte podrás elegir entre dos opciones para ingresar los datos, la primera es dando clic sobre los campos y escribir a través del teclado.
\begin{figure}[H] 
    \centering
        \includegraphics[width=0.85\linewidth]{img/llenar_campos.png}
    \caption{Da clic en los campos y escribe para llenarlos manualmente.}
    \label{fig:llenado_manual}
\end{figure}
La segunda opción es importando los datos desde un archivo de Excel. Para ello, dirígete a la sección de \textbf{Importación desde Excel} que encontrarás hasta arriba de la ventana y haz clic en el botón \textbf{Examinar}. 
A continuación, se abrirá una ventana del Explorador de archivos donde podrás seleccionar el archivo de Excel que contiene los datos de la cotización que deseas importar. Asegúrate de que el archivo esté en formato .xlsx o .xls.

\subsection{Cómo eliminar una cotización}
\subsection{Cómo editar una cotización}
\subsection{Generación del PDF de la cotización}
\subsection{Ejemplo de cotización completa}
%%%%%%%%%%%%%%%%%%%%%%%%%%%%%%%%%%%%%%%%%%%%%%%%%%%%
\section{Resolución a problemas comunes}
\begin{definicion}[No abre correctamente el archivo excel]
    Este apartado aborda problemas comunes que los usuarios pueden encontrar al utilizar el software, 
    proporcionando soluciones prácticas y consejos para resolverlos.
\end{definicion}
\vspace{0.7cm}
\begin{definicion}[No se guardan los cambios realizados]
    Este apartado aborda problemas comunes que los usuarios pueden encontrar al utilizar el software, 
    proporcionando soluciones prácticas y consejos para resolverlos.
\end{definicion}
\vspace{0.7cm}
\begin{definicion}[Error al generar el PDF]
    Este apartado aborda problemas comunes que los usuarios pueden encontrar al utilizar el software, 
    proporcionando soluciones prácticas y consejos para resolverlos.
\end{definicion}
\vspace{0.7cm}
\begin{definicion}[No se muestran las imágenes]
    Este apartado aborda problemas comunes que los usuarios pueden encontrar al utilizar el software, 
    proporcionando soluciones prácticas y consejos para resolverlos.
\end{definicion}
\vspace{0.7cm}
\begin{definicion}[Problemas al ingresar datos]
    Este apartado aborda problemas comunes que los usuarios pueden encontrar al utilizar el software, 
    proporcionando soluciones prácticas y consejos para resolverlos.
\end{definicion}

\subsection{Soporte y contacto}
\pagebreak

\end{document}